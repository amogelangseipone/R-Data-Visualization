% Options for packages loaded elsewhere
\PassOptionsToPackage{unicode}{hyperref}
\PassOptionsToPackage{hyphens}{url}
%
\documentclass[
]{article}
\usepackage{amsmath,amssymb}
\usepackage{iftex}
\ifPDFTeX
  \usepackage[T1]{fontenc}
  \usepackage[utf8]{inputenc}
  \usepackage{textcomp} % provide euro and other symbols
\else % if luatex or xetex
  \usepackage{unicode-math} % this also loads fontspec
  \defaultfontfeatures{Scale=MatchLowercase}
  \defaultfontfeatures[\rmfamily]{Ligatures=TeX,Scale=1}
\fi
\usepackage{lmodern}
\ifPDFTeX\else
  % xetex/luatex font selection
\fi
% Use upquote if available, for straight quotes in verbatim environments
\IfFileExists{upquote.sty}{\usepackage{upquote}}{}
\IfFileExists{microtype.sty}{% use microtype if available
  \usepackage[]{microtype}
  \UseMicrotypeSet[protrusion]{basicmath} % disable protrusion for tt fonts
}{}
\makeatletter
\@ifundefined{KOMAClassName}{% if non-KOMA class
  \IfFileExists{parskip.sty}{%
    \usepackage{parskip}
  }{% else
    \setlength{\parindent}{0pt}
    \setlength{\parskip}{6pt plus 2pt minus 1pt}}
}{% if KOMA class
  \KOMAoptions{parskip=half}}
\makeatother
\usepackage{xcolor}
\usepackage[margin=1in]{geometry}
\usepackage{color}
\usepackage{fancyvrb}
\newcommand{\VerbBar}{|}
\newcommand{\VERB}{\Verb[commandchars=\\\{\}]}
\DefineVerbatimEnvironment{Highlighting}{Verbatim}{commandchars=\\\{\}}
% Add ',fontsize=\small' for more characters per line
\usepackage{framed}
\definecolor{shadecolor}{RGB}{248,248,248}
\newenvironment{Shaded}{\begin{snugshade}}{\end{snugshade}}
\newcommand{\AlertTok}[1]{\textcolor[rgb]{0.94,0.16,0.16}{#1}}
\newcommand{\AnnotationTok}[1]{\textcolor[rgb]{0.56,0.35,0.01}{\textbf{\textit{#1}}}}
\newcommand{\AttributeTok}[1]{\textcolor[rgb]{0.13,0.29,0.53}{#1}}
\newcommand{\BaseNTok}[1]{\textcolor[rgb]{0.00,0.00,0.81}{#1}}
\newcommand{\BuiltInTok}[1]{#1}
\newcommand{\CharTok}[1]{\textcolor[rgb]{0.31,0.60,0.02}{#1}}
\newcommand{\CommentTok}[1]{\textcolor[rgb]{0.56,0.35,0.01}{\textit{#1}}}
\newcommand{\CommentVarTok}[1]{\textcolor[rgb]{0.56,0.35,0.01}{\textbf{\textit{#1}}}}
\newcommand{\ConstantTok}[1]{\textcolor[rgb]{0.56,0.35,0.01}{#1}}
\newcommand{\ControlFlowTok}[1]{\textcolor[rgb]{0.13,0.29,0.53}{\textbf{#1}}}
\newcommand{\DataTypeTok}[1]{\textcolor[rgb]{0.13,0.29,0.53}{#1}}
\newcommand{\DecValTok}[1]{\textcolor[rgb]{0.00,0.00,0.81}{#1}}
\newcommand{\DocumentationTok}[1]{\textcolor[rgb]{0.56,0.35,0.01}{\textbf{\textit{#1}}}}
\newcommand{\ErrorTok}[1]{\textcolor[rgb]{0.64,0.00,0.00}{\textbf{#1}}}
\newcommand{\ExtensionTok}[1]{#1}
\newcommand{\FloatTok}[1]{\textcolor[rgb]{0.00,0.00,0.81}{#1}}
\newcommand{\FunctionTok}[1]{\textcolor[rgb]{0.13,0.29,0.53}{\textbf{#1}}}
\newcommand{\ImportTok}[1]{#1}
\newcommand{\InformationTok}[1]{\textcolor[rgb]{0.56,0.35,0.01}{\textbf{\textit{#1}}}}
\newcommand{\KeywordTok}[1]{\textcolor[rgb]{0.13,0.29,0.53}{\textbf{#1}}}
\newcommand{\NormalTok}[1]{#1}
\newcommand{\OperatorTok}[1]{\textcolor[rgb]{0.81,0.36,0.00}{\textbf{#1}}}
\newcommand{\OtherTok}[1]{\textcolor[rgb]{0.56,0.35,0.01}{#1}}
\newcommand{\PreprocessorTok}[1]{\textcolor[rgb]{0.56,0.35,0.01}{\textit{#1}}}
\newcommand{\RegionMarkerTok}[1]{#1}
\newcommand{\SpecialCharTok}[1]{\textcolor[rgb]{0.81,0.36,0.00}{\textbf{#1}}}
\newcommand{\SpecialStringTok}[1]{\textcolor[rgb]{0.31,0.60,0.02}{#1}}
\newcommand{\StringTok}[1]{\textcolor[rgb]{0.31,0.60,0.02}{#1}}
\newcommand{\VariableTok}[1]{\textcolor[rgb]{0.00,0.00,0.00}{#1}}
\newcommand{\VerbatimStringTok}[1]{\textcolor[rgb]{0.31,0.60,0.02}{#1}}
\newcommand{\WarningTok}[1]{\textcolor[rgb]{0.56,0.35,0.01}{\textbf{\textit{#1}}}}
\usepackage{graphicx}
\makeatletter
\def\maxwidth{\ifdim\Gin@nat@width>\linewidth\linewidth\else\Gin@nat@width\fi}
\def\maxheight{\ifdim\Gin@nat@height>\textheight\textheight\else\Gin@nat@height\fi}
\makeatother
% Scale images if necessary, so that they will not overflow the page
% margins by default, and it is still possible to overwrite the defaults
% using explicit options in \includegraphics[width, height, ...]{}
\setkeys{Gin}{width=\maxwidth,height=\maxheight,keepaspectratio}
% Set default figure placement to htbp
\makeatletter
\def\fps@figure{htbp}
\makeatother
\setlength{\emergencystretch}{3em} % prevent overfull lines
\providecommand{\tightlist}{%
  \setlength{\itemsep}{0pt}\setlength{\parskip}{0pt}}
\setcounter{secnumdepth}{-\maxdimen} % remove section numbering
\ifLuaTeX
  \usepackage{selnolig}  % disable illegal ligatures
\fi
\usepackage{bookmark}
\IfFileExists{xurl.sty}{\usepackage{xurl}}{} % add URL line breaks if available
\urlstyle{same}
\hypersetup{
  pdftitle={R\_Car\_Accidents},
  pdfauthor={Amogelang Alicia Seipone},
  hidelinks,
  pdfcreator={LaTeX via pandoc}}

\title{R\_Car\_Accidents}
\author{Amogelang Alicia Seipone}
\date{2024-11-19}

\begin{document}
\maketitle

\subsection{Introduction}\label{introduction}

As part of my project, I decided to explore the concept of car insurance
pricing using data analysis. Specifically, I wanted to look at how
different factors, like age, gender, and accident history, affect the
risk of getting into an accident and how these factors can influence the
cost of car insurance premiums.

The goal was to create a synthetic dataset that mimics real-world data,
perform some basic analysis on it, and simulate the pricing of car
insurance based on risk scores.

\subsection{Data Generation}\label{data-generation}

To get started, I generated synthetic data that includes: -
\textbf{Age}: Ranging from 18 to 80 years, to simulate a variety of age
groups. - \textbf{Gender}: A random assignment of male or female. -
\textbf{Accident History}: A count of past accidents, ranging from 0 to
5. I used a random number generator to create these values, meaning the
data doesn't represent real individuals but gives us a feel for how
these factors might influence risk.

\begin{Shaded}
\begin{Highlighting}[]
\CommentTok{\# Load Required Libraries}
\FunctionTok{library}\NormalTok{(ggplot2) }\CommentTok{\# For visualization}
\end{Highlighting}
\end{Shaded}

\begin{verbatim}
## Warning: package 'ggplot2' was built under R version 4.3.3
\end{verbatim}

\begin{Shaded}
\begin{Highlighting}[]
\FunctionTok{library}\NormalTok{(dplyr)   }\CommentTok{\# For data manipulation}
\end{Highlighting}
\end{Shaded}

\begin{verbatim}
## Warning: package 'dplyr' was built under R version 4.3.3
\end{verbatim}

\begin{verbatim}
## 
## Attaching package: 'dplyr'
\end{verbatim}

\begin{verbatim}
## The following objects are masked from 'package:stats':
## 
##     filter, lag
\end{verbatim}

\begin{verbatim}
## The following objects are masked from 'package:base':
## 
##     intersect, setdiff, setequal, union
\end{verbatim}

\begin{Shaded}
\begin{Highlighting}[]
\CommentTok{\# Generate Synthetic Data}
\FunctionTok{set.seed}\NormalTok{(}\DecValTok{123}\NormalTok{) }\CommentTok{\# For reproducibility}
\NormalTok{n }\OtherTok{\textless{}{-}} \DecValTok{1000} \CommentTok{\# Number of samples}

\NormalTok{synthetic\_data }\OtherTok{\textless{}{-}} \FunctionTok{data.frame}\NormalTok{(}
  \AttributeTok{ID =} \DecValTok{1}\SpecialCharTok{:}\NormalTok{n,}
  \AttributeTok{Age =} \FunctionTok{sample}\NormalTok{(}\DecValTok{18}\SpecialCharTok{:}\DecValTok{80}\NormalTok{, n, }\AttributeTok{replace =} \ConstantTok{TRUE}\NormalTok{),}
  \AttributeTok{Gender =} \FunctionTok{sample}\NormalTok{(}\FunctionTok{c}\NormalTok{(}\StringTok{"Male"}\NormalTok{, }\StringTok{"Female"}\NormalTok{), n, }\AttributeTok{replace =} \ConstantTok{TRUE}\NormalTok{),}
  \AttributeTok{Accident\_History =} \FunctionTok{sample}\NormalTok{(}\DecValTok{0}\SpecialCharTok{:}\DecValTok{5}\NormalTok{, n, }\AttributeTok{replace =} \ConstantTok{TRUE}\NormalTok{)}
\NormalTok{)}
\end{Highlighting}
\end{Shaded}

\subsection{Risk Scoring}\label{risk-scoring}

The main part of the analysis involves calculating a Risk Score for each
individual based on certain assumptions:

\textbf{Age:} Under 25 years old: Higher risk score (80). 25 to 60 years
old: Medium risk score (50). Over 60 years old: Slightly higher risk
score (70). \textbf{Gender:} Females were assigned a risk score of 20,
and males a risk score of 10. (This is a simple assumption and can be
adjusted based on data or other factors.) \textbf{Accident History:}
Each past accident adds 15 to the risk score.

\begin{Shaded}
\begin{Highlighting}[]
\CommentTok{\# Calculate Risk Scores}
\NormalTok{synthetic\_data }\OtherTok{\textless{}{-}}\NormalTok{ synthetic\_data }\SpecialCharTok{\%\textgreater{}\%}
  \FunctionTok{mutate}\NormalTok{(}
    \AttributeTok{Risk\_Score =} \FunctionTok{case\_when}\NormalTok{(}
\NormalTok{      Age }\SpecialCharTok{\textless{}} \DecValTok{25} \SpecialCharTok{\textasciitilde{}} \DecValTok{80}\NormalTok{,}
\NormalTok{      Age }\SpecialCharTok{\textgreater{}=} \DecValTok{25} \SpecialCharTok{\&}\NormalTok{ Age }\SpecialCharTok{\textless{}=} \DecValTok{60} \SpecialCharTok{\textasciitilde{}} \DecValTok{50}\NormalTok{,}
\NormalTok{      Age }\SpecialCharTok{\textgreater{}} \DecValTok{60} \SpecialCharTok{\textasciitilde{}} \DecValTok{70}\NormalTok{,}
      \ConstantTok{TRUE} \SpecialCharTok{\textasciitilde{}} \DecValTok{0}
\NormalTok{    ) }\SpecialCharTok{+} \FunctionTok{case\_when}\NormalTok{(}
\NormalTok{      Gender }\SpecialCharTok{==} \StringTok{"Female"} \SpecialCharTok{\textasciitilde{}} \DecValTok{20}\NormalTok{,}
\NormalTok{      Gender }\SpecialCharTok{==} \StringTok{"Male"} \SpecialCharTok{\textasciitilde{}} \DecValTok{10}\NormalTok{,}
      \ConstantTok{TRUE} \SpecialCharTok{\textasciitilde{}} \DecValTok{0}
\NormalTok{    ) }\SpecialCharTok{+}\NormalTok{ Accident\_History }\SpecialCharTok{*} \DecValTok{15}
\NormalTok{  )}
\end{Highlighting}
\end{Shaded}

This risk scoring method is a simplified version of how insurance
companies might assess risk, though real insurance models are much more
complex.

\subsection{Simulating Premiums}\label{simulating-premiums}

Once I had the risk scores, I calculated the insurance premium for each
individual:

The premium is determined by multiplying the risk score by 5 and adding
a base amount of 100.

\begin{Shaded}
\begin{Highlighting}[]
\CommentTok{\# Simulate Premiums}
\NormalTok{synthetic\_data }\OtherTok{\textless{}{-}}\NormalTok{ synthetic\_data }\SpecialCharTok{\%\textgreater{}\%}
  \FunctionTok{mutate}\NormalTok{(}\AttributeTok{Premium =}\NormalTok{ Risk\_Score }\SpecialCharTok{*} \DecValTok{5} \SpecialCharTok{+} \DecValTok{100}\NormalTok{)}
\end{Highlighting}
\end{Shaded}

This gives us an idea of how premiums might vary based on risk. For
example, a person with a higher risk score (due to being younger or
having more accidents) would have a higher premium.

\subsection{Data Visualization}\label{data-visualization}

To visualize the data, I created two plots:

\textbf{Risk Score by Age:} A scatter plot that shows how risk scores
vary across different age groups. As expected, younger individuals
(under 25) generally have higher risk scores, while older individuals
(over 60) also show a slight increase.

\textbf{Premium Distribution:} A histogram that shows how premiums are
distributed across the dataset. Most individuals have a premium around
the middle of the range, with fewer people at the extremes (very low or
very high premiums).

\begin{Shaded}
\begin{Highlighting}[]
\CommentTok{\# Visualize Risk Score by Age}
\FunctionTok{ggplot}\NormalTok{(synthetic\_data, }\FunctionTok{aes}\NormalTok{(}\AttributeTok{x =}\NormalTok{ Age, }\AttributeTok{y =}\NormalTok{ Risk\_Score)) }\SpecialCharTok{+}
  \FunctionTok{geom\_point}\NormalTok{(}\AttributeTok{alpha =} \FloatTok{0.5}\NormalTok{) }\SpecialCharTok{+}
  \FunctionTok{labs}\NormalTok{(}\AttributeTok{title =} \StringTok{"Risk Score by Age"}\NormalTok{, }\AttributeTok{x =} \StringTok{"Age"}\NormalTok{, }\AttributeTok{y =} \StringTok{"Risk Score"}\NormalTok{)}
\end{Highlighting}
\end{Shaded}

\includegraphics{Car-Accidents_files/figure-latex/unnamed-chunk-4-1.pdf}

\begin{Shaded}
\begin{Highlighting}[]
\CommentTok{\# Visualize Premium Distribution}
\FunctionTok{ggplot}\NormalTok{(synthetic\_data, }\FunctionTok{aes}\NormalTok{(}\AttributeTok{x =}\NormalTok{ Premium)) }\SpecialCharTok{+}
  \FunctionTok{geom\_histogram}\NormalTok{(}\AttributeTok{bins =} \DecValTok{30}\NormalTok{, }\AttributeTok{fill =} \StringTok{"skyblue"}\NormalTok{, }\AttributeTok{color =} \StringTok{"black"}\NormalTok{) }\SpecialCharTok{+}
  \FunctionTok{labs}\NormalTok{(}\AttributeTok{title =} \StringTok{"Premium Distribution"}\NormalTok{, }\AttributeTok{x =} \StringTok{"Premium"}\NormalTok{, }\AttributeTok{y =} \StringTok{"Frequency"}\NormalTok{)}
\end{Highlighting}
\end{Shaded}

\includegraphics{Car-Accidents_files/figure-latex/unnamed-chunk-4-2.pdf}

\subsection{Results}\label{results}

\textbf{The data shows clear patterns:}

Younger people (especially under 25) tend to have higher risk scores
and, therefore, higher premiums. People with more accidents in their
history also see a rise in premiums. The gender difference in the risk
score is minimal but is included here to show how you could incorporate
different factors into the model.

\begin{Shaded}
\begin{Highlighting}[]
\CommentTok{\# Export Results}
\FunctionTok{write.csv}\NormalTok{(synthetic\_data, }\StringTok{"synthetic\_insurance\_data.csv"}\NormalTok{, }\AttributeTok{row.names =} \ConstantTok{FALSE}\NormalTok{)}

\CommentTok{\# Save plots as images}
\FunctionTok{ggsave}\NormalTok{(}\StringTok{"risk\_score\_by\_age.png"}\NormalTok{, }\AttributeTok{width =} \DecValTok{8}\NormalTok{, }\AttributeTok{height =} \DecValTok{5}\NormalTok{)}
\FunctionTok{ggsave}\NormalTok{(}\StringTok{"premium\_distribution.png"}\NormalTok{, }\AttributeTok{width =} \DecValTok{8}\NormalTok{, }\AttributeTok{height =} \DecValTok{5}\NormalTok{)}
\end{Highlighting}
\end{Shaded}


\end{document}
